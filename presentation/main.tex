\documentclass[xcolor=dvipsnames]{beamer}

\definecolor{commentgreen}{rgb}{0,0.6,0}
\definecolor{cordovagrey}{rgb}{0.2,0.2,0.2}
\definecolor{cordovablue}{rgb}{0.3,0.76,0.89}


\usetheme{Berlin}
\setbeamertemplate{headline} %hide subsection bar at the top
{
	\begin{beamercolorbox}[colsep=1.5pt]{upper separation line head}
	\end{beamercolorbox}
	\begin{beamercolorbox}{section in head/foot}
		\vskip2pt\insertnavigation{\paperwidth}\vskip2pt
	\end{beamercolorbox}
	\begin{beamercolorbox}[colsep=1.5pt]{lower separation line head}
	\end{beamercolorbox}
}

\usefonttheme{professionalfonts}
\usecolortheme[named=cordovablue]{structure}

\setbeamercolor{itemize item}{fg=cordovablue}
\setbeamercolor{itemize subitem}{fg=cordovablue}
\setbeamercolor{enumerate item}{fg=cordovablue}
\setbeamercolor{enumerate subitem}{fg=cordovablue}

\setbeamertemplate{itemize item}[square]
\setbeamertemplate{itemize subitem}[square]
\setbeamertemplate{navigation symbols}{}%remove navigation symbols
\expandafter\def\expandafter\insertshorttitle\expandafter{\insertshorttitle\hfill\insertframenumber\,} %show frame numbers

\usepackage[utf8]{inputenc}
\usepackage[ngerman]{babel} %german language
\usepackage{graphicx} %insert pictures
%\usepackage{svg} %insert vector graphics
\usepackage[export]{adjustbox} %for columns?
\usepackage{listings} %insert code
\lstdefinelanguage{JavaScript}{
	keywords={typeof, new, true, false, catch, function, return, null, catch, switch, var, if, in, while, do, else, case, break},
	keywordstyle=\color{blue}\bfseries,
	ndkeywords={class, export, boolean, throw, implements, import, this},
	ndkeywordstyle=\color{darkgray}\bfseries,
	identifierstyle=\color{black},
	sensitive=false,
	comment=[l]{//},
	morecomment=[s]{/*}{*/},
	commentstyle=\color{purple}\ttfamily,
	stringstyle=\color{green}\ttfamily,
	morestring=[b]',
	morestring=[b]"
}
\lstset{ %configure code listings
	basicstyle=\ttfamily,
	keepspaces=true, 
	numbers=none,  %{none, left, right}
	commentstyle=\color{commentgreen},
	keywordstyle=\color{blue},
	escapeinside={\%*}{*)},
	tabsize=2,
	literate=%
	{Ö}{{\"O}}1
	{Ä}{{\"A}}1
	{Ü}{{\"U}}1
	{ß}{{\ss}}1
	{ü}{{\"u}}1
	{ä}{{\"a}}1
	{ö}{{\"o}}1
}

\usepackage{tikz} %diagrams
\usepackage{verbatim} %diagrams
\usepackage{wasysym} %emojis
\usepackage{hyperref} %links

\title[Entwicklung von Hybrid-Apps am Beispiel von Apache Cordova]{Entwicklung von Hybrid-Apps am Beispiel von Apache Cordova}
%\subtitle{}
\author{Patrick Fehling}
\institute{Hochschule für Technik und Wirtschaft Berlin}
\date{\today}

\begin{document}
\maketitle
\frame{\tableofcontents}


\section{Allgemeines}

\begin{frame}\frametitle{Allgemeine Daten}
	\begin{columns}[t,onlytextwidth]
		\column{.7\textwidth}
		\begin{itemize}
			\item mobile application development framework
			\item Anfängliche Entwicklung durch Nitobi
			\item 2011 Übernahme durch Adobe
			\begin{itemize}
				\item "`PhoneGap"'
				\item "`Apache Cordova"'
			\end{itemize}
			\item Basis für ähnliche Projekte z.B. Ionic, Monaca
			\item Unterstützung durch: Adobe, IBM, Intel, Microsoft
			\item Ziel-OSs: iOS, Android, Blackberry etc.
		\end{itemize}
		\column{.3\textwidth}
		\centering
		
		\includegraphics[width=0.4\textwidth]{pictures/cordova_logo}
		
		\begin{tikzpicture} 
		\node [inner sep=0pt] at (0,0) {\includegraphics[width=2.0cm]{pictures/Pacifica}};
		\draw [white, rounded corners=0.4cm, line width=0.35cm] 
			(current bounding box.north west) -- 
			(current bounding box.north east) --
			(current bounding box.south east) --
			(current bounding box.south west) -- cycle;
		\node[draw] at (2,0) {Pacifica};
		\end{tikzpicture}
		\begin{tikzpicture}
		\node [inner sep=0pt] at (0,0) {\includegraphics[width=2.0cm]{pictures/CleverBaby}};
		\draw [white, rounded corners=0.4cm, line width=0.35cm] 
			(current bounding box.north west) -- 
			(current bounding box.north east) --
			(current bounding box.south east) --
			(current bounding box.south west) -- cycle;
		\node[draw] at (-2,0) {Clever Baby};
		\end{tikzpicture}
	\end{columns}
\end{frame}

\section{Installation \& Build}
\begin{frame}[fragile]\frametitle{Installation - Cordova}
Vorraussetzungen:
\begin{itemize}
	\item Node.js
	\item (optional) Entwicklungsumgebungen für Plattformen
\end{itemize}
\begin{lstlisting}[language=Python]
$ npm install -g cordova
$ cordova create MyApp; cd MyApp
$ cordova platform add <platform>
$ cordova run <platform>
#platform: browser, android, ios...

$ cordova requirements
\end{lstlisting}

\end{frame}

\begin{frame}\frametitle{Installation - Plattformen}
	\begin{columns}[t,onlytextwidth]
		\column{.6\textwidth}
			Android:
			\begin{itemize}
				\item Java SDK (JAVA\_HOME setzen)
				\item Android SDK (ANDROID\_HOME setzen)
				\item gradle (z.B. über Android Studio)
			\end{itemize}
			installieren und der "`PATH"'-Umgebungsvariable hinzufügen\newline
			
		\column{.4\textwidth}
		\centering
		\includegraphics[width=0.9\textwidth,valign=t]{pictures/cordova_device_is_ready}\\
	\end{columns}
	iOS:
	\begin{itemize}
		\item auf OSX: \href{https://cordova.apache.org/docs/en/latest/guide/platforms/ios/index.html}{\textcolor{cordovablue}{iOS Platform Guide}}
		\item auf Windows, Linux: {\LARGE \frownie} oder Adobe PhoneGap Build
	\end{itemize}
\end{frame}

\section{Demo}
\begin{frame}
	\centering
	{\LARGE \textcolor{cordovablue}{Demo}}\\
	\vspace{4ex}
	\href{https://github.com/pa-ling/apache-cordova}{https://github.com/pa-ling/apache-cordova}
\end{frame}

\section{Plugins}
\begin{frame}[fragile]\frametitle{Plugins}
	\begin{columns}[t,onlytextwidth]
		\column{.5\textwidth}
		\begin{lstlisting}[language=Python]
# Hinzufügen
$ cordova plugin add \
		<plugin>
# Liste
$ cordova plugin ls
		\end{lstlisting}
		$\rightarrow$ Überschreiben von Standardfunktionalitäten 
		\\z.B. alert()\\
		$\rightarrow$ Erreichbar über globale Objekte
		\\z.B. device, navigator
		\column{.5\textwidth}
		\begin{itemize}
			\item Kamera
			\item GPS
			\item Vibration
			\item Dialoge
			\item Akkuladestand
			\item Bewegung und Ausrichtung des Geräts
			\item Dateiverwaltung
			\item Push-Notifications
			\item ...
		\end{itemize}
	\end{columns}
\end{frame}

\lstset{numbers=left}
\begin{frame}[fragile]\frametitle{Bluetooth}
	\begin{lstlisting}[language=JavaScript]
ble.scan([], 10, function(device) {
	if ("BLEEnv" === device.name) {
		ble.connect(device.id, function() {
			ble.startNotification(
				device.id,
				SERVICE_UUID,
				CHARACTERISTIC_UUID, 
				function(arrayBuffer) {
					var data = toString(arrayBuffer);
					console.log("Received:" + data);
				});
		});
	}
});
\end{lstlisting}
\end{frame}

\renewcommand{\arraystretch}{1.5}
\section{Nativ vs. Hybrid vs. Web}

\begin{frame}
	\begin{columns}[t,onlytextwidth]
		\begin{tabular}{ l | c | c | c }
			& {\large \textcolor{cordovablue}{Native App}} & {\large \textcolor{cordovablue}{Hybrid App}} & {\large \textcolor{cordovablue}{Web App}} \\
			\hline
			Programmiersprache: & nativ & nativ / web & web \\
			Portabilität: & nein & ja & ja \\
			Gerätezugriff (APIs): & Hoch & Hoch/Mittel & Niedrig \\
			Wissensverfügbarkeit: & Niedrig & Hoch & Hoch \\
			Leistung: & Hoch & Mittel & Mittel \\
			Updates: & Langsam & Mittel & Schnell \\
		\end{tabular}
	\end{columns}
\end{frame}

\begin{frame}
	\begin{columns}[t,onlytextwidth]
		\begin{tabular}{ p{3.2cm} | p{3.2cm} | p{3.2cm} }
			{\large \textcolor{cordovablue}{Native App}} & {\large \textcolor{cordovablue}{Hybrid App}} & {\large \textcolor{cordovablue}{Web App}} \\
			\hline
			Natives Know-How & Web Know-How & Web Know-How \\
			Nur eine Plattform & Kompromiss zwischen Nativ und Web & Schnelle Auslieferung \\
			Viele/aktuelle native Funktionalitäten &  & Schnelle Updates \\
			Hohe Performance &  & mehrere Plattformen \\
		\end{tabular}
	\end{columns}
\end{frame}

\begin{frame}
	\centering
	\textcolor{cordovablue}{{\LARGE Vielen Dank für Ihre Aufmerksamkeit!\\[6ex] Fragen?}}
\end{frame}

\begin{frame}\frametitle{Quellen}
\begin{itemize}
	\item https://cordova.apache.org/
	\item https://ccoenraets.github.io/cordova-tutorial/index.html
	\item https://www.thinkpacifica.com/
	\item https://www.mycleverbaby.com/features.html
	\item https://en.wikipedia.org/wiki/Apache\_Cordova
	\item https://developer.ibm.com/dwblog/2017/what-is-apache-cordova-cross-platform-mobile-application-development/
	\item ftp://public.dhe.ibm.com/software/pdf/mobile-enterprise/WSW14182USEN.pdf
	\item https://www.npmjs.com/package/cordova-plugin-ble-central
\end{itemize}
\end{frame}

\end{document}